\label{shell}

Shellsort is a derivative of insertion sort designed to offer a compromise
between the performance of mergesort and radixsort, and the space efficiency of
selection sort, bubblesort and insertion sort. It was designed by Donald Shell
in 1959, and described in \cite{Shell59}\footnote{Shell describes the former
sorts as ``Merging Pairs'' and ``floating decimal sort'', respectively, and the
latter sorts as ``finding the smallest'', ``interchanging pairs'' and ``sifting'',
respectively.}.
%TODO whats partial sort?

Shellsort begins by considering the array to be made up of $N/2$ subarrays of
length two, where the elements of the subarray are spaced $N/2$ keys apart.
These subarrays are then sorted using insertion sort. The first iteration will
be a simple exchange, but the keys which are exchanged move across the array by
$N/2$ spaces. The sort then considers $N/4$ arrays, who's elements are spaced
$N/4$ keys apart, which are individually sorted by insertion sort, then $N/8$
and so on. The final iteration will be a standard insertion sort, after which
the array will be sorted. A problem with this implementation is that since the
increments are powers of two, keys with odd indices are not compared with keys
with even indices until the final insertion sort. This can lengthen the final
insertion sort's running time to $O(N^2)$.

Since first published, there has been much discussion about the best
\textit{increments} to use in shellsort. Shell's original sort, described above,
has the increments $N/2$, $N/4$, $N/8$, $N/16$ ........ $1$. \cite{Sedgewick96}
discusses many increments used, and provides code for comparing these. Based
on running the \textit{driver} program\footnote{see
http://www.cs.princeton.edu/\textasciitilde{}rs/shell/}, Gonnet's shellsort was
seen to be the fastest on the authors local machine. As a result, this was used
to test the performance of shellsort. This uses increments which are $5/11$ths
of the previous increment.
%TODO do we refer to increments from the top or bottom

Two versions of Gonnet's sort were used. The first was taken directly from the
code presented by Sedgewick, and only slightly modified to fit into the framework used.
The second was modified to reduce the complexity of the loop bounds, and to
remove a sentinel check in the inner loop. Code for these is shown in Figures
\ref{Gonnet Shellsort code} and \ref{Improved Gonnet Shellsort code}.

\code{Gonnet Shellsort code}{old_shellsort.c}
\code{Improved Gonnet Shellsort code}{shellsort.c}

\section{Results}
\subsection{Test parameters}
The increments in Gonnet's shellsort are $N*(5/11)$, $N*(5/11)^2$, $N*(5/11)^3$,
...... $1$.

\subsection{Expected Performance}
Shellsort has been much analysed, but very little is known about the complexity
of the algorithm. The instruction count is expected to be $O(NlogN)$, as this
has been indicated by previous experient. The cycle count is expected to be of
the same order of magnitude as the $O(NlogN)$ sorts for the same reason.
 %TODO find some experiements that showed this.

The cache performance of shellsort should be regular. Each subarray is sorted in
turn without reference to the next. Since each element of a subarray is quite
far from its next element, an entire cache line must be loaded for each key,
which will be used once per iteration of the sort across the subarray. If there
are a large number of subarrays, as during the iterations with large increments,
there will be no spatial reuse, and little temporal reuse. Once the increment is
small enough that each subarray fits within the cache (ie quickly for L2 cache,
and much more slowly for the L1 cache), then temporal reuse will occur. Once an
entire subarray fits within an eighth of the size of the cache (eight being the
number of keys that fit in a cache line), then spatial reuse will occur aswell.
This reuse will only occur within a single iteration of shellsort. Once the next
increment begins, nothing in the cache is likely to be valid, unless the entire
array fits within the cache.

The branch performance of shellsort should be predictable. There are
$log_{5/11}N$ iterations, each with $N$ branch misses\footnote{See Section
\ref{insertion is predictable} for a discussion of the properties of insertion
sort.}. It is expected, therefore, that there will be $log_{5/11}N$ misses per
key.

\subsection{Simulation results}
\plot{shell}{}{}{}{}{}{}

Shellsort's simulation results have the characterist shape and range of an
$O(NlogN)$ sort, albeit, not a very good one. It performs better than a
heapsort, but considerably worse than quicksort. However, it easier to program
than either of these two sort.

Shellsort performs roughly double the number of branches and instructions of
quicksort, and has more than twice as many branch mispredictions and nearly
four times as many cache misses. It's cycle count is over three times that of
quicksort, though it is more than 50 percent faster than the best of the
heapsort. 

\section{Predictor results}

\sixplot{gonnet_shellsort-0.eps}{gonnet_improved_shellsort-0.eps}{}{}{}{}{Shellsorts}

%TODO better writing

In the unimporved version, half the bracnhes are mispredicted. Assuming taken would provide better results. Each key moves only 1.03 spots on average. Once the insertion sort is seperated, we see that this is because the insertion sort tips it in favour of 1 spot, and in actual fact this should be 1.07 spots, on average, though the difference is not very significant. The improved insertion sort is better behaved, moving each key on average two spots. But the prediction ratio is much worse than this. There should be N mispredictions, and instead there are 1.3xN mispredictions. THis is because the counter doesnt really get time to settle properly.


%TODO end here

\section{Future Work}
Only Gonnet's way of producing increments is measured here. Using different
increments and different schemes for generating increments would produce
different results. 
