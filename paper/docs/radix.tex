\label{Radixsort}
%\section{Description}

All other sorts considered in this project are \n{comparison based} sorts.
They compare two keys at a time, and determine the larger and smaller key,
sorting on that basis. Radixsort is a \n{counting based} sort. It works by
separating all the keys into small parts, and processing these parts without
regard for other keys. For radix of two, these small parts are bits, and for a
radix of 10, these parts would be digits. 

The steps to sort by radix are simple. First a radix is chosen, based on the
size of the part to be sorted. For 32 bits integers, a byte is used. This means
the radix is 256, as there are 256 possible byte values. An array of 256
counters is created to count the number of keys with a particular least
significant byte. Then the counters are summed in turn to store the cumulative
count of keys that belong before them in the array. Now each byte can be put in
place by accessing the counter indexed by its least significant byte.

These are written into an auxiliary array, which is written back before the next
iteration using the second least significant byte. Since every key is processed
in order, when the next most significant byte is sorted, the order of the less
significant bytes are maintained, so that once the most significant byte is
sorted, the entire array is.

This type of radixsort is called \n{Least Significant Digit} radixsort.
Other implementations are possible, and several improvements also exist, but as
radixsort is included here for comparative purposes, these are not implemented.
They are, however, discussed in Section \ref{radix future work}.

Note that this method is only suitable for unsigned ints. Because a signed int's
sign is contained in the most significant bit, this sort would place negative
numbers after the positive numbers.

\subsection{Cache Radixsort}
Many improvements exist to radixsort, and some are suggested by Sedgewick and
LaMarca. Some of these improvements help to lower cache misses and instruction
count, both of which slow down radixsort significantly. LaMarca suggested
that while the current byte is being put in place, the counting of the next byte
can be done. This increases the memory requirements, but this is not
significant.

The first improvement is to reduce the number of runs across the array per
step.  In the \n{base radixsort}, presented above, there is one step to count,
one to copy, and one to write back (which accesses both arrays). LaMarca
suggests counting for the next step in the previous count. In fact, all the
counting can be done in a single pass at the start, instead of four passes at
various stages, decreasing instruction count and cache misses. The downside is
the fourfold increase in memory required for the arrays of counts, with no
benefits.

The step to copy the array back can also be removed. In mergesort, copying back
to the source array was changed to copying back and forth between the source and 
auxiliary arrays. The same could be potentially done with radixsort, without any
increase in complexity. The number of steps is likely to be even, and is known
at compile time, so the number of times the copy step is done could be reduced
to zero. If the number of steps is odd, which is unlikely when sorting
integers, then one copy at the end would be required. This is not needed in the
implementation presented here.

Cache radixsort should traverse the array only nine times: Once at the start to
count the least significant bits; six times as it copies using the three least
significant bytes (counting the next least significant byte at the same time),
and a final two as it copies using the most significant byte.

\section{Results}
\subsection{Test parameters}
Keys are to be divided into segments 8 bits long, resulting in four sorting
steps to completely sort the array. The radix used is therefore $2^8 = 256$.

\subsection{Expected Performance}
Radixsort is expected to have very good performance, compared to comparison
based sorts. This is due to that fact that radixsort is $O(N)$; for an array as
small as 4096 records, it involves 12 times fewer steps than the $O(NlogN)$
sorts.

The cache performance of base radixsort is not expected to be good, however.
Quicksort and mergesort, for example, have very good temporal locality. Keys are
used repeatedly before being swapped out of the cache. In radixsort, however,
every key is used just once each time it is in the cache. When the array is larger than
the size of the cache, every key access will be a miss as a result, and every
new byte being sorted involves four passes over the array. In addition, if the
auxiliary array is pathologically aligned, then the copy back into the array
can result in thrashing.

The number of iterations across the array is constant, however, proportional to
$(bitsPerInt/log_2(Radix))$. For a 32 bit integer, this is four steps and each step
has four passes over the array, for a total of 16 passes over the array.  With
eight keys per cache line, there should be about two misses per key, when the
array doesn't fit into the array, though this depends on the placing of the
auxiliary array.

The number of branch prediction misses should be exceptionally low. In
comparison based sorts, a branch due to a comparison can not be predicted except
in exceptional circumstances\footnote{See Section \ref{insertion is
predictable}.}. As a result, only flow control branches occur in radixsort, and
they are all predictable.

Due to the reduced copying, cache radixsort should have a lower instruction
count and fewer cache misses that base radixsort. There should be nine passes
over the array, leading to just over one miss per key, as opposed to two misses
per key with base radixsort. The complexity and branch prediction results
should remain the same.

\subsection{Simulation results}
\plot{radix}{}{}{}{}{}{}

The instruction count of base radixsort, shown in Figure \ref{radix instructions} is
very low. It is lower than even quicksort for the smallest set, and it
flat-lines immediately (after amortisation of the loop overhead). The number of
cache misses in Figure \ref{radix level 2 misses} is exactly the same, until
the array no longer fits into the cache. The number of misses then leaps up to
2.5 per key, which is roughly what was predicted.

Cache radixsort performs as predicted aswell. By sorting keys between two
arrays instead of copying a partially sorted array back, the instruction count
has been reduced by more than 10\%, and the number of level 2 cache misses
have been more than halved, to just over one miss per key.

The number of branch misses, shown in Figure \ref{radix branch misses} was also
extremely low. Despite 10 branches per key, there was less than .08 of a branch
miss per key. The number of branches itself is quite low, as several of the
sorts considered had nearly that number of misses. However, it is not
unexpected. The number of branches reduces when copying back is removed, and
while this reduces the number of mispredictions, there are so few to begin with
that this result is not significant.

With the low number of branch misses, a much better overall performance is
expected than is achieved. When in the cache, it performs only slightly worse
than quicksort, but once the data set grows larger, the gap widens.  However,
the results, shown in Figure \ref{radix cycles}, indicate that radixsort will
never increase beyond 350 cycles per key, meaning that for larger sets it will
perform significantly better than quicksort. Cache radixsort never increases
beyond 300 cycles, though it has relatively more cycles when the arrays are
smaller. 

The most curious result is that all simulations show that cache radixsort
performing better than the base radixsort, yet the Pentium 4 results show that
base radixsort performs better for smaller arrays.

LaMarca's radixsort results are quite different to those presented here. His
instruction count graph has a logarithmic shape: it begins very high and sharply
falls before flattening out. By contrast, the graph here is almost completely
flat. A similar thing happens in both cache miss graphs: a sharp dip in
LaMarca's results, and a flat-line with a slight slope in these results. Also,
in both cases, once the array no longer fits in the cache, the number of misses
jumps to around 2.5, where they stay. 

The cycles per key in both cases are unsurprising based on instruction count and
cache miss results: LaMarca's have a sharp fall before rising slightly once the
data set no longer fits in the cache; the result presented here flat-line before rising
slightly for the same reason. While both results agree for large data sets, they do not
for smaller sets.

\section{Future Work}
\label{radix future work}
Combining other sorts with radixsort could also be possible. An optimisation to
mergesort was to do a k-way merge at the end, but this dramatically increased
the instruction count. Using a type of radixsort here would be complicated, but
may be possible. Alternatively, sorting inside the cache using radixsort is far
faster than merging. The k-way merge could then be performed at the end. In
practice, though, radixsort is far faster than mergesort, has a lower
instruction count, and less cache misses. While these steps could improve
mergesort, they are more likely to slow radixsort, with no advantage.

Radixsort could also be faster than multi-quicksort's initial step. Replacing
this with a single radix step would split the array into 256 containers, each of
which could be quicksorted. Alternatively, a smaller radix, based on the size of
the array, could be generated at run time. Using LaMarca's analysis,
$C/3$ containers need to be generated. With $N$ as the number of keys in the
array, and $C$ is the number of keys which fit in the level 2 cache, the radix
can be calculated with the formula:
$$2^{\lceil{}log_2(\frac{3N}{CacheSize})\rceil{}}$$
This will only work for random, evenly distributed keys. If all the keys only
vary in their least significant bits, then this will have no effect.

The radixsort presented above is a Least Significant Digit radixsort. A Most
Significant Digit radixsort behaves differently. Once the initial sort is done, each
bin in recursed into, and continually sorted in this manner. This results in a
higher instruction count, and increased stack usage due to the recursion.
However, there should be a large increase in temporal reuse due to this, which
may make up for it.
