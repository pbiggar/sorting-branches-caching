Most major sorting algorithms in computer science date to the 50s and 60s.
Radixsort was written in 1954, quicksort in 1962, and mergesort has been traced
back to the 1930s. Modern processors meanwhile have features which were not
thought of when these sorts were written: branch predictors, for example, were
not created until the 1970s, and advanced branch prediction techniques were
not used in consumer processors until very recently. It is hardly surprising,
then, that sorting algorithms rarely take these architectural features into
account.

Papers considering sorting performance typically analyse instruction count.
LaMarca and Ladner \cite{LaMarca96} break this trend by discussing the cache
implications of several algorithms, including sorts. This project attempts to
investigate these claims and to test and analyse the high performance sorts
based on them.

LaMarca devotes a chapter of his thesis to sorting. A chapter is spent
discussing implicit heaps, which are later used in heapsort. In
\cite{LaMarca99}, radixsort was added for comparison. Radixsort is added here
for the same purpose. For comparison, elementary sorts - insertion sort,
selection sort, bubblesort and shakersort - are also included here.

In addition to repeating LaMarca's work, this project also discusses hardware
branch prediction techniques and their effects on sorting. When algorithms are
altered to improve their performance, it is important to observe if any change
in the rate of branch prediction misses occurs.

Over the course of the cache-conscious algorithm improvements, branch prediction
results are generated in order to observe the side-effects of the
cache-conscious improvements on the branch predictors, and whether they affect
different branch predictors in different ways. 

Several important results are discovered during the course of the project. The
branch prediction rates of insertion sort and selection sort, for example, and
the difference in performance between binary searches and sequential searches due
to branch misses. Several steps to remove instructions from standard algorithms
are devised, such as double tiling mergesort, and removing a sentinel and bounds
check from heapsort.

Chapter 2 provides background information on caches and branch predictors. It
also discusses the theory of sorting, and conventions used in the algorithms in
this project.

Chapter 3 discusses the method and framework used in this project, including the
tools used and the reasons for using them.

Chapter 4 discusses the elementary sorts and their implementations. Results are
included for later comparison.

Chapters 5 through 7 discuss base algorithms and cache-conscious improvements
for heapsort, mergesort and quicksort. Results are presented and explained, and
branch prediction implications of the changes are discussed.

Chapter 8 presents a radixsort implementation and discusses its performance in
the areas of instruction count, cache-consciousness and branch prediction.

Chapter 9 presents conclusions and contributions developed over the course of
this project.
