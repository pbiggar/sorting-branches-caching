\section{Bugs}
This section lists some of the bugs that exist in the remaining code. In a
project of limited timescale, it is not possible to remove every possible bug
from the code. These bugs are listed here in order that it is possible to
continue on this work with minimal difficutly.

\section{Algorithmic}

\label{bug1}
In multi-quicksort, the results include the sentinel check which should have
been removed before the results were measured. This adds one branch per \cc{N /
THRESHOLD} keys , and the same number of extra instructions. In the same
algorithm, the old number of 1018 is the number of keys in a block of a list.
The effects of this are not significant. Also, the check for the sentinel was
not removed, though the effect of this is minor.

\label{bug2}
In several of the mergesorts, it is possible that an extra merge step is being
taken. This has not been properly analysed, due to time restrictions. It is,
however, obvious from the level 2 miss graph of all the merges, that this
does not affect base mergesort, but it does affect both double mergesorts, and
the tiled mergesort. It is uncertain if multi mergesort is affected. The problem
stems from the end condition of the merge. It should end once the size of the next
merge is equal the size of the array, rather than greater than. However, the
algorithms do not properly merge when this is the case.

\label{bug4}
Double tiled mergesort and double multi-mergesort were optimized to sort in the
level 1 cache first. The number of keys that fit into the cache was fixed at
2048; this number should have been half that, since the sort is out of place.
This probably resulted in thrashing, especially since the arrays were aligned.
Removing this would improve the sorts.

\section{Results}
\label{bug3}
To reduce the costs of creating the array of random numbers, this time was
measured and subtracted from the results. While this works as expected for
instruction count, the effects on cache misses is less clear. Subtracting level
2 cache misses incurred during filling removes compulsory misses from the
equation. For shorter lists, the compulsory misses are completely removed.
For longer lists, those greater than the size of the cache, compulsory misses
come back, since the keys at the start of the array are flushed from the cache
by the time they begin to be sorted. LaMarca's work centers around
reducing conflict misses, however, so this is not necessarily a problem, just
something to be aware of.
