\label{radix}
%\section{Description}

All other sorts considered in this project are \n{comparison based} sorts.
They compare two keys at a time, and determine the larger and smaller key,
sorting on that basis. Radixsort is a \n{counting based} sort. It works by
separating the keys into small parts, and processing these parts without
regard for other keys. For radix of two, these small parts are bits, and for a
radix of 10, these parts would be digits. 

The steps to sort by radix are simple. First a radix is chosen, based on the
size of the part to be sorted. For 32 bit integers, a byte is used. This means
the radix is 256, as there are 256 possible byte values. An array of 256
counters is created to count the number of keys with a particular least
significant byte. The counters are then summed in turn to store the cumulative
count of keys with a smaller least significant byte. Now each key can be put in
place by accessing the counter indexed by its least significant byte, after
which that counter is incremented.

The keys are written into an auxiliary array, which is written back to the initial
array before the next iteration. The process is then repeated using the second
least significant byte.  Since every key is processed in order, when the next
most significant byte is sorted, the order of the less significant bytes are
maintained, so that once the most significant byte is sorted, the entire array
is.

\section{Base Radixsort}

This type of radixsort is called \n{Least Significant Digit} radixsort.
Other implementations are possible, including a \n{most significant bit}
version, which behaves as a divide and conquer algorithm. These are discussed
in Section \ref{radix future work}.

Note that this method is only suitable for \cc{unsigned int}s. Because a
\cc{signed int}'s sign is contained in the most significant bit, this sort would
place negative numbers after the positive numbers.

\section{Cache Radixsort}
Many improvements exist to radixsort, and some are suggested by Sedgewick and
LaMarca. Some of these improvements help to lower cache misses and instruction
count, both of which slow down radixsort significantly. LaMarca suggests an
improvement by Friend\footnote{See \ref{TODO friend[8]}.} that while the current
byte is being put in place, the counting of the next byte can be done. This
increases the memory requirements, but this is not significant.

The first improvement is to reduce the number of runs across the array per
step.  In the \n{base radixsort}, presented above, there is one step to count,
one to copy, and one to write back (which accesses both arrays). We take
LaMarca's suggestion and begin the counting of the next least significant byte
in the previous iteration.

In fact, all the counting could be done in a single pass at the start, instead
of four passes at various stages, decreasing instruction count and cache misses.
The downside is the fourfold increase in memory required for the arrays of
counts, with no benefits.

The step to copy the array back can also be removed. In mergesort, copying back
to the source array was changed to copying back and forth between the source and
auxiliary arrays. The same could be done with radixsort, without any increase in
complexity. The number of steps is likely to be even, and depends on the radix.
It is known at compile time, so the number of times the copy step is done could
be reduced to zero since we choose an even radix. If the number of steps is odd,
which is unlikely when sorting integers, then one copy at the end would be
required. This is not needed in the implementation presented here.

Memory-tuned radixsort should traverse the array only nine times: Once at the
start to count the least significant bytes; six times (once per array per pass)
as it copies using the three least significant bytes (counting the next least
significant byte at the same time), and a final two as it copies using the most
significant byte.


\subsection{Aligned Memory-tuned Radixsort}
In Section \ref{tiled mergesort}, mergesort is improved by aligning its array
with reference to it auxiliary array. Specifically, extra memory is allocated
for the auxiliary array, which is then offset so that the start of the two
arrays are exactly half the length of the cache apart. We further added an
alignment for the level 1 cache (see Section \ref{double-aligned tiled
mergesort}) to significantly reduce level 1 misses at the expense of a small
number of level 2 misses. We adapt this strategy here, and call it Aligned
Memory-tuned Radixsort.


\section{Results}
\subsection{Test Parameters}
Keys are to be divided into segments 8 bits long, resulting in four sorting
steps to completely sort the array. The radix used is therefore $2^8 = 256$.

\subsection{Expected Performance}
Radixsort is expected to have very good performance, compared to comparison
based sorts. This is due to that fact that radixsort is $O(N)$; for an array as
small as 4096 records, it involves 12 times fewer steps than the $O(NlogN)$
sorts. For an array as large as 4194304, an $(O(NlogN)$ sort will take 22 times
as many steps to complete. However, it is not expected that radixsort will
perform many times better than quicksort, the reigning champion of $O(NlogN)$
sorts. A `step' for quicksort is as simple as a comparison each step and a
possible exchange every few steps.  For radixsort, a step comprises being
counted four times and copied four times, with the counting involving several
layers of indirection, and causing a complete iteration across the array at the
start. LaMarca's results show the instruction count of quicksort and radixsort
being in the same order of magnitude, and we expect the same.

The cache performance of base radixsort is not expected to be good, however.
Quicksort and mergesort have very good temporal locality. Keys are
used repeatedly before being swapped out of the cache. In radixsort, however,
every key is used just once each time it is in the cache. When the array is larger than
the size of the cache, every key access will be a miss as a result, and every
new byte being sorted involves four passes over the array. In addition, if the
auxiliary array is pathologically aligned, then the copy back into the array
can result in thrashing.

The thrashing can also occur in aligned radixsort. Unlike mergesort, we cannot
guarantee that the keys being considered are at the same positions in both
arrays. In fact, we have no idea where is being accessed in the source array, so
it is unlikely that the alignment will have a great effect.

The number of iterations across the array is constant, however, proportional to
$(bitsPerInt/log_2(Radix))$. For a 32 bit integer and a radix of 256, this is
four steps and each step has four passes over the array, for a total of 16
passes over the array.  With eight keys per cache line, there should be about
two misses per key, when the array doesn't fit into the array, though this
depends on the placing of the auxiliary array.

The number of branch prediction misses should be exceptionally low. In
comparison based sorts, a branch due to a comparison can not be predicted except
in exceptional circumstances\footnote{See Section \ref{insertion is
predictable}.}. Radixsort has no comparative branches at all. As a result, only
flow control branches occur in radixsort, and they are all predictable.

Due to the reduced copying, memory-tuned radixsort should have a lower instruction
count and fewer cache misses that base radixsort. There should be nine passes
over the array, leading to just over one miss per key, as opposed to two misses
per key with base radixsort. The complexity and branch prediction results
should remain the same. The instruction count should, however, reduce by nearly
a half to $\frac{9}{16}^ths$ of base radixsort.

Based on LaMarca's results, showing quicksort to be faster than radixsort by a
small amount, we expect memory-tuned radixsort to run faster than quicksort.

\subsection{Simulation Results}
\plot{radix}{}{}{}{}{}{}

The instruction count of base radixsort, shown in Figure \ref{radix instructions} is
very low. It is lower than even quicksort for the smallest set, and it
flat-lines immediately (after slight amortisation of the loop overhead). The
number of cache misses in Figure \ref{radix level 2 misses} is only slightly
more than quicksort - due to being out-of-place - until the array no longer fits
into the cache. The number of misses then leaps up to 2.5 per key, which is
slightly more than was predicted. The extra is likely due to the unpredictable
way that the source array is accessed as the keys are copied to the auxiliary
array.

Memory-tuned radixsort performs as predicted as well. By sorting keys between two
arrays instead of copying a partially sorted array back, the instruction count
has been reduced by more than 10\%, and the number of level 2 cache misses
have been halved, to one and a quarter miss per key. This is exactly
as expected, due to there being nine passes over the array, and a one miss
per pass per cache line (coming to slightly under one and a quarter. The extra
is again probably due to the unpredictable order of the copying).

The number of misses while the array fits in the cache is obvious: there
are two compulsory misses per key - one for each array - which divided by the
keys per cache line results in exactly one miss in four (although our graph
obscures this, due to the initial subtraction of compulsory misses).

Aligned memory-tuned radixsort performs exactly the same as memory-tuned radixsort in both
level 1 and level 2 misses, indicating that the alignment, and the associated
memory overhead, are not necessary.

The number of branch misses, shown in Figure \ref{radix branch misses} was also
extremely low. Despite three branches per key, there was less than .02 of a branch
miss per key. The number of branches itself is quite low, as several of the
sorts considered had nearly that number of misses. However, it is not
unexpected. The number of branches reduces when copying back is removed, and
while this reduces the number of mispredictions, there are so few to begin with
that this result is not significant.

Again we see that the bimodal branch predictor performs better than the two
level adaptive. In this case it is contrary to expectations, however, as
there are no comparative branches. However, there are so few misses that these
results do not show anything; in the worst case there are slightly over 200
misses in total.

With the low number of cache misses, low instruction count and negligible branch
mispredictions, it is no surprise that memory-tuned radixsort performs as well as it
does. As shown in Figure \ref{radix cycles}, memory-tuned radixsort's performance is
almost linear, with only a slight increase when the arrays no longer fit in the
cache.  Even then, the increase in cache misses are more than made up for by the
lack of branch mispredictions, and quicksort ends up taking almost 50\% percent
longer to sort more than 4 million keys.

LaMarca's radixsort results are quite different to those presented here. His
instruction count graph has a logarithmic shape: it begins very high and sharply
falls before flattening out. By contrast, the graph here is almost completely
flat. He shows a similar result in both cache miss graphs: a sharp dip, followed
by a nearly flat-line. Also, in both cases, once the array no longer fits in the
cache, the number of misses jumps to around 2.5, where they stay.

However, LaMarca's conclusion - that radixsort is not as fast as quicksort - can
be easily seen from the base radixsort results. Once an radixsort optimised for
the cache is presented, these no longer hold, and the lack of branch
misprediction results in a new fastest sort.

\section{Future Work}
\label{radix future work}
Combining other sorts with radixsort could be possible.  Radixsort should be be
faster than multi-quicksort's initial step. Replacing this with a single radix
step would split the array into 256 containers, each of which could be
quicksorted. Alternatively, a smaller radix, based on the size of the array,
could be generated at run time. Using LaMarca's analysis, $C/3$ containers need
to be generated. With $N$ as the number of keys in the array, and $C$ is the
number of keys which fit in the level 2 cache, the radix can be calculated with
the formula:
$$2^{\lceil{}log_2(\frac{3N}{CacheSize})\rceil{}}$$
This will only work for random, evenly distributed keys. If all the keys only
vary in their least significant bits, then this will have no effect.

The radixsort presented above is a Least Significant Digit radixsort. A Most
Significant Digit radixsort behaves differently. Once the initial sort is done,
each bin is recursed into, and continually sorted in this manner. This results
in a higher instruction count, and increased stack usage due to the recursion.
However, there should be a large increase in temporal reuse due to this, which
may make up for it.
