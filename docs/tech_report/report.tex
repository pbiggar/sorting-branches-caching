\documentclass{report}
\usepackage{ifthen}
\usepackage{varioref}
\usepackage{listings}
\usepackage{color}
\usepackage{graphicx}
\usepackage{subfigure}
\usepackage{float}
\usepackage{lgrind}

\usepackage[intlimits]{amsmath}
\usepackage{eufrak}
\usepackage[mathscr]{euscript}
\usepackage{afterpage}

% widens the margins for the graphs
\newenvironment{changemargin}{%
 \begin{list}{}{%
\setlength{\textwidth}{\paperwidth}
\setlength{\textheight}{\paperheight}
%\setlength{\oddsidemargin}{-1in}
%\setlength{\evensidemargin}{-1in}
\setlength{\topmargin}{-1in}
\setlength{\topsep}{-1in}%
	\setlength{\leftmargin}{-1.5in}%
	\setlength{\rightmargin}{-1.5in}%
	\setlength{\listparindent}{\parindent}%
	\setlength{\itemindent}{\parindent}%
	\setlength{\parsep}{\parskip}%
 }%
\centering%
\item[]%
}{\end{list}}

\newenvironment{spacedpars}{%for the abstract
\setlength{\parskip}{1em}
\setlength{\parindent}{0ex}
}

%\cc{x} is used for
% - unix commans or programs: gcc, sim-bpred, sim-cache, espresso, li, simple,
%   sim-fast, diff, valgrind, cachegrind, driver
% - code or keywords: less, Item, exch, if, while, unsigned int, int, qsort,
%   goto, do-while, THRESHOLD, partition, malloc, shellsort, improved shellsort
% - file names: main.c
% - #defines: UNIT_MAX
\newcommand{\cc}[1]{\texttt{#1}}

%\n{x} is used for
% - proper names: Atom, SimpleScalar, Papi, GnuPlot, ValGrind, glibc, PapiEx
% - technical names: unsigned integer, less-than, greater-than, in-place, out-of
%   place, Abstract Data Type, records, cache hierarchy, latency, hit, miss,
%   compulsory, cold-start, capacity, conflict, cache line, cache block, offset,
%   tag, virtaul addresses, bus address, temporal locality, locality, spatial
%   locality, taken, not taken, flow control branch, comparaitve branch, static,
%   semi-static, dynamic, 1-bit, bimodal, two-level adaptive, heap, children,
%   implicit heap, heapifying, sifting, fix-up, fix-down, sift-up, sift-down,
%   Repeated-adds, Remove-min, d-heap, fanout, natural, straight, unrolled,
%   Memory management unit, tiling, k-way merge, trashing divide and conquer,
%   multi-way partition, comparision based, counting based, increments
% - web addresses: http://www.gnuplot.com, http://valgrind.kde.org
% - sort names, when highlighted: algorithm S, algorithm N
% - branch predictor names: 'Creation, child promotion', etc
\newcommand{\n}[1]{\textit{#1}}

% to include a file as code
\newcommand{\code}[2]{\begin{figure}\lstinputlisting{code/#2}\caption{#1}\label{#1}\end{figure}}

% used for scaling graphs
\newcommand{\myscale}{1.2}

% plots graphs across 3 pages
\newcommand{\plot}[7]{
% 3 figures first, then 3 page for them
\afterpage{
\setcounter{subfigure}{0}

\thispagestyle{empty}
\clearpage
\enlargethispage{14em}
\vspace*{-8em}
\begin{figure}[H]
\begin{changemargin}
\subfigure[Cycles per key - this was measured on a Pentium 4 using hardware performance counters. #2]
{\label{#1 cycles}\includegraphics[scale=\myscale]{plots/#1_-_cycles.eps}}
\subfigure[Instructions per key - this was simulated using SimpleScalar \cc{sim-cache}. #3]	
{\label{#1 instructions}\includegraphics[scale=\myscale]{plots/#1_-_Instructions.eps}}
\end{changemargin}
\vspace*{7em}
\caption{Simulated instruction count and empiric cycle count for #1sort}
\label{Simulated instruction count and empiric cycle count for #1sort}
\end{figure}
\newpage

\thispagestyle{empty}
\clearpage
\enlargethispage{14em}
\vspace*{-8em}
\begin{figure}[H]
\begin{changemargin}
\subfigure[Level 1 cache misses per key - this was simulated using SimpleScalar
\cc{sim-cache}, simulating an 8KB data cache with a 32-byte cache line and
separate instruction cache. #4] {\label{#1 level 1
misses}\includegraphics[scale=\myscale]{plots/#1_-_Level_1_misses.eps}}\\
\subfigure[Level 2 cache misses per key - this was simulated using SimpleScalar
\cc{sim-cache}, simulating a 2MB data and instruction cache and a 32-byte cache
line. #5]
{\label{#1 level 2 misses}\includegraphics[scale=\myscale]{plots/#1_-_Level_2_misses.eps}}\\
\end{changemargin}
\vspace*{7em}
\caption{Cache simulation results for #1sort}
\label{Cache simulation results for #1sort}
\end{figure}
\newpage

\thispagestyle{empty}
\clearpage
\enlargethispage{14em}
\vspace*{-7em}
\begin{figure}[H]
\begin{changemargin}
\centering
\subfigure[Branches per key - this was simulated using \cc{sim-bpred}. #6]
{\label{#1 branches}\includegraphics[scale=\myscale]{plots/#1_-_Branches.eps}}
\subfigure[Branches misses per key - this was simulated using \cc{sim-bpred},
with bimodal and two-level adaptive predictors. The simulated two-level adaptive
predictor used a 10-bit branch history register which was XOR-ed with the
program counter. #7]
{\label{#1 branch misses}\includegraphics[scale=\myscale]{plots/#1_-_Branch_misses.eps}}
\end{changemargin}
\vspace*{6em}
\caption{Branch simulation results for #1sort}
\label{Branch simulation results for #1sort}
\end{figure}
}}


% the scaling for the smaller graphs
\newcommand{\smallscale}{.74}

% plots 6 images on 1 page
\newcommand{\smallplot}[2]{
\setcounter{subfigure}{0}
\thispagestyle{empty}
\clearpage
\enlargethispage{14em}
\vspace*{-5em}
\begin{figure}[H]
\begin{changemargin}
\subfigure[Cycles per key] {\includegraphics[scale=0.74]{plots/#1-_cycles.eps}}
\subfigure[Instructions per key] {\includegraphics[scale=0.74]{plots/#1_-_Instructions.eps}}
\subfigure[Level 1 cache misses per key]	{\includegraphics[scale=0.74]{plots/#1_-_Level_1_misses.eps}}
\subfigure[Level 2 cache misses per key]	{\label{#2 level 2 misses}\includegraphics[scale=0.74]{plots/#1_-_Level_2_misses.eps}}
\subfigure[Branches per key]				{\includegraphics[scale=0.74]{plots/#1_-_Branches.eps}}
\subfigure[Branch misses per key]			{\includegraphics[scale=0.74]{plots/#1_-_Branch_misses.eps}}
\end{changemargin}
\vspace*{6em}
\caption{Simulation results for #2}
\label{Simulation results for #2}
\end{figure}
\newpage
}

% plots 6 or less images on a page, using the counters
\newcommand{\sixplot}[7]{
\afterpage{
\setcounter{subfigure}{0}
\thispagestyle{empty}
\clearpage
\enlargethispage{14em}
\vspace*{-5em}
\begin{figure}[H]
\begin{changemargin}
\subfigure[]{\label{#7-1}\includegraphics[scale=0.74]{plots/counter_#1}}
\ifthenelse{\equal{#2}{}}{}{\subfigure[]{\label{#7-2}\includegraphics[scale=0.74]{plots/counter_#2}}}
\ifthenelse{\equal{#3}{}}{}{\subfigure[]{\label{#7-3}\includegraphics[scale=0.74]{plots/counter_#3}}}
\ifthenelse{\equal{#4}{}}{}{\subfigure[]{\label{#7-4}\includegraphics[scale=0.74]{plots/counter_#4}}}
\ifthenelse{\equal{#5}{}}{}{\subfigure[]{\label{#7-5}\includegraphics[scale=0.74]{plots/counter_#5}}}
\ifthenelse{\equal{#6}{}}{}{\subfigure[]{\label{#7-6}\includegraphics[scale=0.74]{plots/counter_#6}}}
\end{changemargin}
\vspace*{6em}
\caption[Branch prediction performance for #7]{Branch prediction performance for
#7. All graphs use the same scale. Above each column is the total number of
branches and the percentage of correctly predicted branches for a simulated
bimodal predictor.}
\label{Branch prediction performance for #7}
\end{figure}
\newpage
}}

% settings for the code listings
\lstset{language=c,tabsize=3,basicstyle=\small,showspaces=false,showtabs=false,showstringspaces=false,stepnumber=1,numberstyle=\small,frame=tblr}

% i actually have no idea what this does, and no memory of writing it
\long\def\symbolfootnote[#1]#2{\begingroup\def\thefootnote{\fnsymbol{footnote}}\footnote[#1]{#2}\endgroup}

\topmargin 0.2cm
\textwidth 16cm
\textheight 21cm
\oddsidemargin 0.0cm
\evensidemargin 0.0cm

\bibliographystyle{these}

% how to hyphenate words which arent standard
\hyphenation{La-Marca}

%TODO make the page wider

% ----------------------------------------------------------------------------------
% END OF PREAMBLE
% ----------------------------------------------------------------------------------
\begin{document}

\begin{titlepage}
       \begin{center}
				\vspace*{40mm}
                \Huge Sorting in the Presence of Branch Prediction and Caches\\
                \vspace{1em}
                \Large Fast Sorting on Modern Computers\\
                \vspace{2em}
                \Large Paul Biggar \hspace{1cm} David Gregg\\
				\vspace{2em}
				\large Technical Report TCD-CS-2005-57\\
                \large Department of Computer Science,\\
				\large University of Dublin, Trinity College,\\
				\large Dublin 2, Ireland.\\
				\vspace{2em}
				\large August, 2005\\
        \end{center}
\end{titlepage}

\begin{spacedpars}
\begin{abstract}
Sorting is one of the most important and studied problems in computer science.
Many good algorithms exist which offer various trade-offs in efficiency,
simplicity and memory use. However most of these algorithms were discovered
decades ago at a time when computer architectures were much simpler than today.
Branch prediction and cache memories are two developments in computer
architecture that have a particularly large impact on the performance of sorting
algorithms.

This report describes a study of the behaviour of sorting algorithms on branch
predictors and caches. Our work on branch prediction is almost entirely new, and
finds a number of important results. In particular we show that insertion sort
causes the fewest branch mispredictions of any comparison-based algorithm, that
optimizations - such as the choice of the pivot in quicksort - can have a large
impact on the predictability of branches, and that advanced two-level branch
predictors are usually worse at predicting branches in sorting algorithms than
simpler branch predictors. In many cases it is possible to draw links between
classical theoretical analyses of algorithms and their branch prediction
behaviour.

The other main work described in this report is an analysis of the behaviour of
sorting algorithms on modern caches. Over the last decade there has been
considerable interest in optimizing sorting algorithms to reduce the number of
cache misses. We experimentally study the cache performance of both classical
sorting algorithms, and a variety of cache-optimized algorithms proposed by
LaMarca and Ladner. Our experiments cover a much wider range of algorithms than
other work, including the $O(N^2)$ sorts, radixsort and shellsort, all within a
single framework. We discover a number of new results, particularly relating to
the branch prediction behaviour of cache-optimized sorts.

We also developed a number of other improvements to the algorithms, such as
removing the need for a sentinel in classical heapsort. Overall, we found that a
cache-optimized radixsort was the fastest sort in our study; the absence of
comparison branches means that the algorithm causes almost no branch
mispredictions.

\end{abstract}
\end{spacedpars}

\tableofcontents
\listoffigures

\setlength{\parskip}{1em}
\setlength{\parindent}{0ex}

\chapter{Introduction}

\input intro.tex

\chapter{Background Information}

\input background.tex

\chapter{Tools and Method}

\input method.tex

\chapter{Elementary Sorts}

\input elementary.tex

\chapter{Heapsort}

\input heap.tex

\chapter{Mergesort}

\input merge.tex

\chapter{Quicksort}

\input quick.tex

\chapter{Radixsort}

\input radix.tex

\chapter{Shellsort}

\input shell.tex

\chapter{Conclusions}

\input conclusions.tex

\appendix

\chapter{Simulation Result Listing}
\input result_list.tex

%\input software_predictor.tex

% the only remaining thing is the compulsory cache subtraction.
\chapter{Bug List}
\input bugs.tex

\bibliography{report} % param is the name of the .bib file
\nocite{*}

\end{document}
