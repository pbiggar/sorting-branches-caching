\label{intro}
Most major sorting algorithms in computer science date to the 50s and 60s.
Radixsort was written in 1954, quicksort in 1962, and mergesort has been traced
back to the 1930s. Modern processors meanwhile have features which were not
thought of when these sorts were written: branch predictors, for example, were
not created until the 1970s, and advanced branch prediction techniques were not
used in consumer processors until very recently. It is hardly surprising, then,
that sorting algorithms rarely take these architectural features into account.

Papers considering sorting performance typically analyse instruction count.
LaMarca and Ladner break this trend in \cite{LaMarca96} by discussing the cache
implications of several algorithms, including sorts. This report investigates
these claims and tests and analyses high performance sorts based upon their
work.

LaMarca devotes a chapter of his thesis to sorting. Another chapter is spent
discussing implicit heaps, which are later used in heapsort. In
\cite{LaMarca99}, radixsort was added for comparison. Radixsort and shellsort
are added here for the same purpose. For comparison, elementary sorts -
insertion sort, selection sort, bubblesort and shakersort - are also included.


In addition to repeating LaMarca's work, this report also discusses hardware
branch prediction techniques and their effects on sorting. When algorithms are
altered to improve their performance, it is important to observe if any change
in the rate of branch prediction misses occurs.

Over the course of the cache-conscious algorithm improvements, branch prediction
results are generated in order to observe the side-effects of the
cache-conscious improvements on the branch predictors, and whether they affect
different branch predictors in different ways. 

Several important discovered results are reported here. The branch prediction
rates of insertion sort and selection sort, for example, and the difference in
performance between binary searches and sequential searches due to branch
misses. Several steps to remove instructions from standard algorithms are
devised, such as double tiling and aligning mergesort, removing a sentinel and
bounds check from heapsort, and adding a copy-back step to radixsort.


Chapter \ref{background} provides background information on caches and branch
predictors. It also discusses the theory of sorting, and conventions used in the
algorithms in this report.

Chapter \ref{method} discusses the method and framework used in this research,
including the tools used and the reasons for using them.

Chapter \ref{ordernsquared}  discusses the elementary sorts and their
implementations. Results are included for later comparison.

Chapters \ref{heap} through \ref{radix} discuss base algorithms and
cache-conscious improvements for heapsort, mergesort, quicksort and radixsort.
Results are presented and explained, and branch prediction implications of the
changes are discussed.

Chapter \ref{shell} presents two shellsort implementations and discusses their
performance in the areas of instruction count, cache-consciousness and branch
prediction.

Chapter \ref{conclusions} presents conclusions, summarises results from previous
chapters, compares our results with LaMarca's, presents a list of cache and
branch prediction results and improvements, discusses the fastest sort in our
tests and lists contributions made by this research.
